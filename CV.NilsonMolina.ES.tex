\documentclass[10.5pt,letterpaper]{article}

\usepackage[utf8]{inputenc}
\usepackage[T1]{fontenc}
\usepackage{lmodern}
\usepackage[margin=0.7in]{geometry}
\usepackage[hidelinks]{hyperref}
\usepackage{enumitem}
\setlist[itemize]{noitemsep, topsep=0pt}
\renewcommand{\baselinestretch}{1.15}
\setlength{\parindent}{0pt}

\begin{document}
	
	\begin{center}
		{\LARGE \textbf{Nilson Aldair Molina Rengifo}}\\[4pt]
		\textbf{Desarrollador de Software Full-Stack}\\[6pt]
		\href{mailto:nilsonmolina@outlook.com}{nilsonmolina@outlook.com} • +57 320 903 0250 • 
		\href{https://linkedin.com/in/nilsonmolina}{linkedin.com/in/nilsonmolina}
	\end{center}
	
	\hrule
	\vspace{0.5em}
	
	% ===== Resumen Profesional =====
	\section*{Resumen Profesional}
	Desarrollador de software full-stack con más de 5 años de experiencia en el diseño y desarrollo de aplicaciones web utilizando Laravel, Vue.js, APIs RESTful, Docker y bases de datos relacionales (MySQL, PostgreSQL). Competente en la creación de sistemas backend eficientes, despliegue de servicios en contenedores y escritura de código mantenible con enfoque en pruebas. Experiencia en entornos ágiles, migración de sistemas heredados y mejora de la confiabilidad del software mediante pruebas y arquitectura limpia. Enfocado en generar valor medible para usuarios y equipos.
	
	% ===== Experiencia Profesional =====
	\section*{Experiencia Profesional}
	
	\textbf{Consultor de Software} - Amaris Consulting, Bogotá, Colombia
	\hfill
	\textit{Feb 2025 – Abr 2025}
	\begin{itemize}
		\item Desarrollé más de 8 nuevas funcionalidades orientadas al usuario en un entorno ágil y de alta demanda.
		\item Resolví más de 20 tickets de soporte y reduje el backlog de errores en aproximadamente 30\% durante el primer mes.
		\item Realicé mantenimiento y optimización de módulos backend clave, mejorando los tiempos de respuesta en un 20\%.
	\end{itemize}
	
	\textbf{Ingeniero de Desarrollo} - Pacaribe S.A. E.S.P., Bogotá, Colombia
	\hfill
	\textit{Jul 2024 – Ene 2025}
	\begin{itemize}
		\item Implementé funcionalidades centrales en aplicaciones internas basadas en Laravel, impactando el flujo de trabajo diario de más de 50 colaboradores.
		\item Escribí y mantuve más de 40 pruebas PHPUnit, alcanzando una cobertura cercana al 80\% y aumentando la confianza en los despliegues.
		\item Elaboré documentación técnica y colaboré con analistas para refinar y clarificar requerimientos.
	\end{itemize}
	
	\textbf{Desarrollador Laravel (Freelance)} - FrenziApp
	\hfill
	\textit{Feb 2024 – Abr 2024}
	\begin{itemize}
		\item Refactoricé la lógica del backend y reestructuré los endpoints de la API, reduciendo los tiempos de respuesta en un 35\%.
		\item Implementé Docker para estandarizar el entorno de desarrollo, reduciendo en un 50\% el tiempo de incorporación de nuevos desarrolladores.
		\item Escribí pruebas de integración y mejoré la confiabilidad del backend mediante una mejor validación y manejo de errores.
	\end{itemize}
	
	\textbf{Desarrollador Laravel (Freelance)} — Ink Master Colombia
	\hfill
	\textit{Ago 2023 – Feb 2024}
	\begin{itemize}
		\item Diseñé un sistema CRM personalizado con módulos para inventario, clientes, proveedores y facturación, utilizado diariamente por más de 10 empleados.
		\item Reduje el tiempo de generación de facturas en un 70\% mediante automatización y generación de PDF.
		\item Implementé soporte para Docker y pruebas automatizadas para optimizar los despliegues y asegurar la estabilidad del sistema.
	\end{itemize}
	
	\textbf{Programador} — Solvo Global, Bogotá, Colombia
	\hfill
	\textit{Sep 2022 – Jun 2023}
	\begin{itemize}
		\item Analicé requerimientos de clientes y desarrollé herramientas utilizadas por más de 5,000 usuarios.
		\item Reduje los errores reportados en un 45\% mediante desarrollo guiado por pruebas y revisiones de código.
		\item Entregué cuatro módulos reutilizables que fueron adoptados por otros equipos dentro de la compañía.
	\end{itemize}
	
	\textbf{Desarrollador I} — THOMAS MTI, Bogotá, Colombia
	\hfill
	\textit{Abr 2021 – Nov 2021}
	\begin{itemize}
		\item Migré una aplicación empresarial de Delphi a Angular/.NET, reduciendo los costos de mantenimiento en un 50\%.
		\item Colaboré en un equipo de tres personas para rediseñar la interfaz y mejorar la usabilidad con base en retroalimentación interna.
		\item Desarrollé scripts de migración de datos complejos, garantizando la integridad del 100\% de la información.
	\end{itemize}
	
	\textbf{Analista de Desarrollo de Software} — CLTech, Bogotá, Colombia
	\hfill
	\textit{Jul 2019 – Feb 2021}
	\begin{itemize}
		\item Diseñé e implementé software conforme a los estándares ISO 15189 para laboratorios clínicos.
		\item Dividí un sistema monolítico en una arquitectura basada en APIs RESTful, mejorando la escalabilidad y reduciendo la complejidad del código.
		\item Lideré mejoras en el módulo de CRM que aumentaron la eficiencia operativa en tres departamentos.
	\end{itemize}
	
	\textbf{Desarrollador de Software} — Recaudo Bogotá S.A.S., Bogotá, Colombia  
	\hfill
	\textit{Jul 2018 – Ene 2019}
	\begin{itemize}
		\item Desarrollé un sistema de programación de turnos para más de 1,500 empleados, optimizando la planificación y asignación de recursos.
		\item Automatizé la generación de certificados y estadísticas, reduciendo la carga manual en un 40\%.
	\end{itemize}
	
	% ===== Educación =====
	\section*{Educación}
	
	\textbf{Ingeniería de Sistemas} \hfill \textit{Finalización estimada 2026} \\  
	Universidad Central — Bogotá, Colombia \\  
	Enfoque en ingeniería de software, arquitectura de sistemas y gestión de información.
	
	\textbf{Tecnólogo en Análisis y Desarrollo de Sistemas de Información} \hfill \textit{2019} \\  
	Servicio Nacional de Aprendizaje (SENA) — Bogotá, Colombia \\  
	Formación práctica en diseño de bases de datos, desarrollo web y despliegue de proyectos.
	
	\textbf{Técnico en Programación de Software} \hfill \textit{2016} \\  
	Servicio Nacional de Aprendizaje (SENA) — Bogotá, Colombia \\  
	Formación introductoria en lógica de programación, algoritmos y aplicaciones de escritorio.
	
	% ===== Formación y Certificaciones =====
	\section*{Formación y Certificaciones}
	\begin{itemize}
		\item Construcción de Bases de Datos con MySQL \hfill SENA (2025)
		\item Python: Variables y Estructuras de Control \hfill SENA (2024)
		\item Despliegue de Aplicaciones y Servicios con Docker \hfill SENA (2023)
		\item Desarrollo Web con PHP \hfill SENA (2018)
	\end{itemize}
	
	% ===== Habilidades Técnicas =====
	\section*{Habilidades Técnicas}
	\textbf{Lenguajes:} PHP, JavaScript, SQL, Python (básico)\\
	\textbf{Frameworks y Librerías:} Laravel, Vue.js, Bootstrap, jQuery\\
	\textbf{Bases de Datos:} MySQL, PostgreSQL\\
	\textbf{Herramientas y DevOps:} Git, Docker, Postman, PHPUnit, Swagger\\
	\textbf{Prácticas:} APIs RESTful, MVC, Pruebas Unitarias, Desarrollo Ágil, Dockerización
	
\end{document}
