\documentclass[9pt,letterpaper]{extarticle}

\usepackage[utf8]{inputenc}
\usepackage[T1]{fontenc}
\usepackage{lmodern}
\usepackage[margin=0.6in]{geometry}
\usepackage[hidelinks]{hyperref}
\usepackage{enumitem}
\usepackage{titlesec}

\setlist[itemize]{itemsep=0pt, topsep=0pt, parsep=0pt, partopsep=0pt, leftmargin=*}
\setlength{\parskip}{0.5em}
\setlength{\parindent}{0pt}

\renewcommand{\baselinestretch}{1.09}

\titlespacing*{\section}{0pt}{0.5em}{0.25em}

\begin{document}
	
	\begin{center}
		{\LARGE \textbf{Nilson Aldair Molina Rengifo}}
		
		\textbf{Software Developer (Backend / Full‑Stack)}
		
		\href{mailto:nilsonmolina@outlook.com}{nilsonmolina@outlook.com} •
		\href{https://wa.me/573209030250}{(+57) 320 903 0250} •
		\href{https://linkedin.com/in/nilsonmolina}{linkedin.com/in/nilsonmolina}
	\end{center}

	\hrule

	\section*{Resumen Profesional}
		Desarrollador de Software con más de 6 años de experiencia construyendo, estabilizando y mejorando sistemas backend y APIs en producción dentro de entornos críticos para el negocio. Sólida experiencia en contextos predominantemente backend y full-stack, trabajando de manera cercana con equipos de producto, frontend y áreas de negocio para entregar funcionalidades confiables y escalables de punta a punta.
		
		Experiencia asumiendo la responsabilidad de sistemas y funcionalidades desde el diseño hasta producción, incluyendo decisiones de arquitectura, modelado de datos, optimización de rendimiento y estabilización en producción. Capacidad comprobada para operar de forma efectiva en bases de código complejas, mejorar sistemas heredados y entregar resultados bajo restricciones operativas reales.
		
		Capaz de asumir responsabilidad en roles de alto impacto, contribuir a la dirección técnica y mantener altos estándares de ingeniería mientras equilibra velocidad, calidad y necesidades del negocio.
	
	\section*{Experiencia Profesional}
	
	\textbf{Ingeniero de Software (Freelance)} - {PC Pronto}
	\hfill \textit{Jul 2025 - Presente}
	\begin{itemize}
		\item Diseñé y entregué servicios backend que soportan una plataforma de comercio electrónico con tráfico real en producción, con énfasis en escalabilidad, consistencia de datos y tolerancia a fallos.
		\item Construí y mantuve APIs RESTful integrando Shopify con un sistema backend centralizado utilizando Node.js, AWS y PostgreSQL.
		\item Implementé sincronización bidireccional de datos con SuiteCRM, definiendo reglas de validación y lógica de resolución de conflictos entre sistemas.
		\item Asumí la responsabilidad de funcionalidades completas desde la arquitectura y la implementación hasta el despliegue y la estabilización posterior al lanzamiento.
		\item Colaboré directamente con stakeholders del negocio para traducir necesidades operativas en soluciones técnicas mantenibles.
	\end{itemize}
	
	\textbf{Consultor de Software} - {Amaris Consulting}
	\hfill \textit{Feb 2025 - Abr 2025}
	\begin{itemize}
		\item Contribuí al mantenimiento y evolución de un sistema backend monolítico de gran escala que gestiona operaciones de reparación de dispositivos electrónicos en múltiples ubicaciones.
		\item Implementé funcionalidades backend y ajustes de flujos de trabajo utilizando PHP (Yii, Laravel) en procesos críticos para el negocio.
		\item Diagnostiqué y resolví incidentes recurrentes en producción, mejorando la estabilidad del sistema y la confiabilidad operativa.
		\item Mejoré el rendimiento mediante refactorizaciones puntuales y optimización de consultas a base de datos (PostgreSQL).
		\item Participé activamente en revisiones de código y ceremonias ágiles dentro de un entorno Scrum.
	\end{itemize}
	
	\textbf{Ingeniero de Desarrollo} - {Pacaribe S.A. E.S.P.}
	\hfill \textit{Jul 2024 - Ene 2025}
	\begin{itemize}
		\item Trabajé en múltiples sistemas orientados al backend que centralizaban datos operativos, regulatorios y analíticos de la organización, utilizando principalmente Laravel, Node.js, Vue.js, PostgreSQL y Oracle (solo lectura).
		\item Contribuí a la evolución de una plataforma monolítica central basada en Laravel y Vue que agregaba información proveniente de aplicaciones internas y procesos ETL, funcionando como el principal sistema operativo de la compañía.
		\item Diseñé e implementé mejoras backend en servicios adicionales basados en Node.js que soportaban lógica de negocio regulatoria, incorporando modelos de datos estructurados y acceso mediante ORM para mejorar la mantenibilidad y la consistencia de la información.
		\item Mejoré el rendimiento de base de datos mediante la identificación de consultas lentas o sin índices, la eliminación de patrones de acceso innecesarios y la optimización de consultas PostgreSQL utilizadas por flujos de alto tráfico.
		\item Implementé procesamiento asíncrono utilizando colas de Laravel, comandos personalizados y trabajos en segundo plano para manejar de forma confiable operaciones de larga duración y alto volumen de datos.
		\item Desarrollé un módulo de planeación de tareas dentro del sistema central, soportando tareas jerárquicas, fechas límite y alertas automáticas por correo electrónico, mejorando la coordinación y la visibilidad operativa entre equipos.
		\item Asumí responsabilidad por la entrega y calidad de las soluciones dentro de los ciclos de sprint, con evaluación basada en la efectividad, estabilidad y mantenibilidad de las funcionalidades entregadas.
		\item Brindé seguimiento técnico y continuidad operativa a un proceso ETL existente basado en .NET durante una transición de gestión, asegurando su ejecución diaria ininterrumpida y el control de incidencias.
	\end{itemize}
	
	\textbf{Desarrollador Laravel (Freelance)} - {FrenziApp}
	\hfill \textit{Feb 2024 - Abr 2024}
	\begin{itemize}
		\item Fui responsable de una API REST basada en Laravel que soportaba aplicaciones nativas Android e iOS para una plataforma de reserva de taxis.
		\item Refactoricé el código backend para mejorar la claridad, la mantenibilidad y la consistencia en el diseño de las APIs.
		\item Optimicé las respuestas de la API y la estructura de los payloads, mejorando significativamente los tiempos de respuesta.
		\item Introduje documentación Swagger / OpenAPI e implementé pruebas unitarias (PHPUnit) para la lógica de negocio principal.
	\end{itemize}
	
	\textbf{Desarrollador de Software Full-Stack (Freelance)} - {Ink Master Colombia}
	\hfill \textit{Ago 2023 - Feb 2024}
	\begin{itemize}
		\item Diseñé e implementé un sistema interno centralizado de gestión utilizando Laravel, Vue.js y PostgreSQL para soportar las operaciones principales del negocio.
		\item Desarrollé módulos backend para la gestión de clientes, inventario, productos, proveedores y ventas, con foco en la consistencia de datos y flujos de negocio confiables.
		\item Asumí la responsabilidad técnica completa del sistema, desde el modelado de datos y la lógica backend hasta la integración frontend y el despliegue inicial.
		\item Desplegué el sistema en un entorno basado en AWS durante su lanzamiento inicial, asegurando su preparación para producción.
		\item Centralicé datos operativos previamente fragmentados en una única plataforma, mejorando la visibilidad y la coordinación entre las áreas del negocio.
		\item Mejoré los procesos comerciales y operativos de la empresa al proporcionar acceso estructurado a información de clientes, inventario y proveedores.
	\end{itemize}
	
	\textbf{Desarrollador de Software Full-Stack} - {Solvo Global}
	\hfill \textit{Sep 2022 – Jun 2023}
	\begin{itemize}
		\item Trabajé como ingeniero integrado para un cliente con sede en Estados Unidos, contribuyendo al desarrollo de funcionalidades con enfoque backend y al mantenimiento del sistema.
		\item Implementé y ajusté la lógica de la aplicación en colaboración con equipos distribuidos, siguiendo procesos de entrega definidos por el cliente.
		\item Brindé soporte a flujos de trabajo en producción, resolviendo incidencias y asegurando la continuidad operativa.
		\item Trabajé bajo un modelo de ingeniería tercerizada orientado al cliente, adaptándome a estándares externos de entrega, tiempos y expectativas de calidad.
	\end{itemize}
	
	\section*{Experiencia Profesional (Etapa Temprana de Carrera - Consolidada)}
	
	\textbf{Desarrollo Independiente}
	\hfill \textit{Dic 2021 - Ago 2022}
	\begin{itemize}
		\item Participé en desarrollo de software de forma independiente, fortalecimiento técnico y resolución de problemas en contextos tipo freelance.
		\item Consolidé fundamentos de backend, diseño de bases de datos y arquitectura de APIs mediante proyectos prácticos y aprendizaje autodirigido.
	\end{itemize}
	
	\textbf{Desarrollador de Software} - {CLTech, THOMAS MTI, Recaudo Bogotá S.A.S.}
	\hfill \textit{Jul 2018 - Nov 2021}
	\begin{itemize}
		\item Contribuí a sistemas empresariales con fuerte enfoque backend en sectores de servicios públicos, salud y gestión de talento humano.
		\item Brindé soporte a plataformas monolíticas y a iniciativas tempranas de APIs RESTful, adquiriendo experiencia práctica en modularización backend.
		\item Participé en migraciones de datos, procesos de modernización de sistemas legados y mejoras de herramientas internas.
		\item Construí una base sólida en entornos productivos, manejo de datos y colaboración con equipos multidisciplinarios.
	\end{itemize}
	
	
	\section*{Education}
	
		\textbf{Ingeniería de Sistemas y Computación}
		\textit{(en curso)}
		\hfill {Universidad Central (CO)}
		
		\textbf{Tecnólogo en Análisis y Desarrollo de Sistemas de Información}
		\hfill {Servicio Nacional de Aprendizaje (SENA)}
		
		\textbf{Técnico Profesional en Programación de Software}
		\hfill {Servicio Nacional de Aprendizaje (SENA)}
	
	\section*{Formación y Certificaciones}
	\begin{itemize}
		\item Despliegue de aplicaciones y servicios en contenedores docker 
		\hfill 2023
		
		\item Variables y estructuras de control en python
		\hfill 2024
		
		\item Construcción de bases de datos con mysql
		\hfill 2025
		
		\item Algoritmo de agrupamiento no supervisado k-means con python
		\hfill 2025
		
		\item Apropiación de los conceptos en ciberseguridad
		\hfill 2025
		
		\item Aplicación del marco de trabajo scrum para proyectos de desarrollo de software
		\hfill 2025
		
	\end{itemize}
	
\end{document}
