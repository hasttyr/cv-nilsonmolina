\documentclass[10pt,letterpaper]{article}

\usepackage[utf8]{inputenc}
\usepackage[T1]{fontenc}
\usepackage{lmodern}
\usepackage[margin=0.6in]{geometry}
\usepackage[hidelinks]{hyperref}
\usepackage{enumitem}
\usepackage{titlesec}

\setlist[itemize]{noitemsep, topsep=0pt, leftmargin=*}
\setlength{\parindent}{0pt}

\renewcommand{\baselinestretch}{1.08}

\titlespacing*{\section}{0pt}{0.7em}{0.35em}

\begin{document}
	
	\begin{center}
		{\LARGE \textbf{Nilson Aldair Molina Rengifo}}\\[4pt]
		\textbf{Desarrollador de Software Full Stack}\\[6pt]
		\href{mailto:nilsonmolina@outlook.com}{nilsonmolina@outlook.com} • 
		\href{https://wa.me/573209030250}{(+57) 320 903 0250} • 
		\href{https://linkedin.com/in/nilsonmolina}{linkedin.com/in/nilsonmolina}
	\end{center}
	
	\hrule
	
	% ===== Professional Summary =====
	\section*{Resumen Profesional}
	Ingeniero de software con fuerte enfoque en backend y experiencia trabajando en sistemas en producción en distintos niveles de madurez. La mayor parte de mi trayectoria ha estado centrada en plataformas basadas en Laravel, donde he trabajado con sistemas intensivos en datos, APIs consumidas por aplicaciones móviles y herramientas internas que soportan flujos operativos. Suelo trabajar cerca de los límites del sistema, tomando decisiones técnicas pragmáticas relacionadas con calidad de datos, rendimiento, confiabilidad y contratos de API. Con el tiempo, mis responsabilidades han evolucionado de ejecución guiada a asumir propiedad técnica completa como colaborador individual.
	
	% ===== Professional Experience =====
	\section*{Experiencia Profesional}
	
	\textbf{Ingeniero de Software (Freelance)} — {PC Pronto}, {Bogotá, Colombia}
	\hfill \textit{Jul 2025 – Actualidad}
	\begin{itemize}
		\item Diseño e implementación de una capa de integración entre SuiteCRM y Shopify para centralizar los datos de ventas retail en un sistema backend principal.
		\item Definición de flujos de sincronización de datos, reglas de validación y lógica de transformación para garantizar la consistencia entre sistemas operativos y centralizados.
		\item Traducción de requerimientos de negocio de alto nivel en soluciones técnicamente viables, aclarando alcance, restricciones y trade-offs en iniciativas de automatización.
		\item Trabajo con servicios y APIs alojados en la nube para soportar intercambios de datos confiables entre plataformas.
	\end{itemize}
	
	\textbf{Consultor de Software} – {Amaris Consulting}, {Bogotá, Colombia}
	\hfill \textit{Feb 2025 – Abr 2025}
	\begin{itemize}
		\item Participación en el diseño e implementación de funcionalidades backend en un entorno ágil de alta velocidad, traduciendo requerimientos de negocio ambiguos en soluciones escalables.
		\item Identificación y corrección de problemas sistémicos en el backend, reduciendo bugs recurrentes en producción y disminuyendo el volumen total de incidentes en aproximadamente un 30\%.
		\item Optimización de módulos críticos mediante refactorización y mejoras en consultas, logrando una reducción del 20\% en los tiempos de respuesta.
	\end{itemize}
	
	\textbf{Ingeniero de Desarrollo} – {Pacaribe S.A. E.S.P.}, {Bogotá, Colombia}
	\hfill \textit{Jul 2024 – Ene 2025}
	\begin{itemize}
		\item Diseño e implementación de funcionalidades backend en sistemas internos basados en Laravel que manejan grandes volúmenes de datos heterogéneos, con foco en calidad de datos, eficiencia en el acceso y estabilidad del sistema.
		\item Liderazgo de estrategias de depuración y validación de datos para evitar que información inconsistente o de baja calidad llegara a las aplicaciones frontend, mejorando el rendimiento y reduciendo reprocesos.
		\item Optimización de consultas y modelos de datos para soportar recuperación de información a gran escala, mejorando significativamente los tiempos de respuesta y la eficiencia general del sistema.
	\end{itemize}
	
	\textbf{Desarrollador Laravel (Freelance)} – {FrenziApp}
	\hfill \textit{Feb 2024 – Abr 2024}
	\begin{itemize}
		\item Asunción de decisiones técnicas backend para un sistema Laravel existente consumido por dos aplicaciones móviles nativas, alineando el diseño y comportamiento de las APIs con los requerimientos del cliente móvil.
		\item Refactorización y estabilización de la lógica backend y de los endpoints de API, logrando una mejora aproximada del 35\% en los tiempos de respuesta.
		\item Diseño e implementación de nuevos módulos backend para cálculo de rutas y estimación de costos, abordando la lógica central del dominio en una plataforma de transporte.
		\item Definición y documentación de contratos backend utilizando Swagger, reduciendo fricción y ambigüedad en la integración con los equipos móviles.
		\item Fortalecimiento de la confiabilidad del backend mediante pruebas de integración, manejo estructurado de errores y validaciones de entrada más estrictas.
	\end{itemize}
	
	\textbf{Desarrollador Laravel (Freelance)} — {Ink Master Colombia}
	\hfill \textit{Ago 2023 – Feb 2024}
	\begin{itemize}
		\item Diseño y construcción de un sistema CRM monolítico utilizando Laravel y Vue.js, asumiendo responsabilidad total sobre la arquitectura backend y frontend, modelado de datos y decisiones de despliegue.
		\item Implementación de módulos clave del negocio (clientes, inventario/productos, ventas, proveedores), estructurando la lógica de dominio para soportar los flujos operativos diarios.
		\item Automatización de la generación de recibos de venta y registros transaccionales en PDF, reduciendo el tiempo de procesamiento de documentos en aproximadamente un 70\%.
		\item Implementación de despliegues contenerizados y automatización de pruebas para asegurar consistencia de entornos, confiabilidad y mantenibilidad.
	\end{itemize}
	
	\textbf{Programador de Computadores} — {Solvo Global}, {Bogotá, Colombia}
	\hfill \textit{Sep 2022 – Jun 2023}
	\begin{itemize}
		\item Participación en una plataforma monolítica basada en Laravel y Vue.js utilizada para la programación de turnos de enfermería en Estados Unidos, dentro de un dominio de salud con altos requerimientos de confiabilidad.
		\item Trabajo con un motor de flujos de trabajo orientado a eventos (Camunda) integrado al backend para manejar procesos asíncronos y mejorar la resiliencia del sistema ante fallos.
		\item Mejora de la calidad del código y estabilidad del sistema mediante desarrollo guiado por pruebas y revisiones estructuradas, contribuyendo a una reducción del 45\% en defectos reportados.
		\item Desarrollo de módulos backend reutilizables que posteriormente fueron adoptados por otros equipos, promoviendo consistencia y reutilización en la plataforma.
	\end{itemize}
	
	\textbf{Developer I} — {THOMAS MTI}, {Bogotá, Colombia}
	\hfill \textit{Abr 2021 – Nov 2021}
	\begin{itemize}
		\item Contribución a la migración de una aplicación empresarial de Delphi a Angular/.NET, implementando componentes asignados bajo guía senior y ayudando a reducir costos de mantenimiento a largo plazo.
		\item Colaboración en un equipo pequeño para mejorar la experiencia de usuario basada en retroalimentación interna, apoyando la usabilidad de flujos heredados.
		\item Desarrollo y ejecución de scripts de migración de datos, asegurando integridad y consistencia durante las transiciones del sistema.
	\end{itemize}
	
	\textbf{Analista de Desarrollo de Software} — {CLTech}, {Bogotá, Colombia}
	\hfill \textit{Jul 2019 – Feb 2021}
	\begin{itemize}
		\item Implementación de componentes de software para sistemas de laboratorio clínico alineados con la norma ISO 15189, evolucionando de responsabilidades junior a semi-senior bajo acompañamiento técnico.
		\item Contribución a la descomposición progresiva de una aplicación monolítica hacia APIs REST, adquiriendo experiencia práctica en diseño backend modular.
		\item Entrega de mejoras en módulos tipo CRM que soportaban flujos operativos de múltiples áreas.
	\end{itemize}
	
	\textbf{Desarrollador de Software} — {Recaudo Bogotá S.A.S.}, {Bogotá, Colombia}
	\hfill \textit{Jul 2018 – Ene 2019}
	\begin{itemize}
		\item Desarrollo y despliegue de componentes de software en producción para un sistema de programación de personal expuesto a usuarios reales del sector público.
		\item Automatización de la generación de certificados operativos y estadísticas básicas, reduciendo la carga administrativa manual en aproximadamente un 40\%.
	\end{itemize}
	
	% ===== Education =====
	\section*{Educación}
	
	\textbf{Ingeniería de Sistemas} \\  
	Universidad Central — Bogotá, Colombia \hfill \textit{En curso}  
	
	\textbf{Tecnólogo en Análisis y Desarrollo de Sistemas de Información} \\
	Servicio Nacional de Aprendizaje (SENA) — Bogotá, Colombia \hfill \textit{2019}
	
	\textbf{Técnico en Programación de Software} \\
	Servicio Nacional de Aprendizaje (SENA) — Bogotá, Colombia \hfill \textit{2016}
	
	% ===== Training & Certifications =====
	\section*{Formación y Certificaciones}
	\begin{itemize}
		\item Despliegue de Aplicaciones y Servicios con Docker \hfill SENA (2023)
		\item Construcción de Bases de Datos con MySQL \hfill SENA (2025)
		\item Desarrollo de Software Ágil con Scrum \hfill SENA (2025)
		\item Fundamentos de Ciberseguridad \hfill SENA (2025)
		\item Python para Backend y Procesamiento de Datos \hfill SENA (2024)
		\item Aprendizaje Automático No Supervisado: K-Means con Python \hfill SENA (2025)
	\end{itemize}
	
	% ===== Technical Skills =====
	\section*{Habilidades Técnicas}
	
	Backend: PHP, Laravel, diseño de APIs REST, monolitos modulares \\
	Datos: MySQL, PostgreSQL, modelado de datos, optimización de consultas \\
	Infraestructura: Docker, AWS (EC2, S3), GitLab CI, Bitbucket Pipelines \\
	Calidad: PHPUnit, pruebas de integración, refactorización de sistemas legacy \\
	Forma de trabajo: Agile (Scrum), documentación técnica, colaboración entre equipos
	
\end{document}
